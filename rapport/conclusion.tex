\section {Conclusion}
Ce projet nous a permis de pratiquer de nombreuses compétences importantes pour les futurs informaticiens que nous sommes
telles que le travail de recherche et la collaboration.\\

Nous avons dû appronfondir nos connaissances en programmation C pour répondre à des besoins de compréhension des formats 
de fichier mais aussi pour prendre les meilleures décisions concernant l'implémentation du code.
Dans un premier temps, nous nous sommes efforcés d'afficher les fichiers afin d'en maitriser les différentes sections
et contraintes qui y sont liées.\\

Nous avons ensuite commencé à coder selon une technique reconnue pour être simple et efficace qui se base sur le LSB évoquée
plus haut. Il s'agit donc de travailler sur les bits de poids faible ce qui crée une différence imperceptible à l'oeil humain.
D'abord dans le bitmap, pour des questions de simplicité, puis dans le gif.\\

Nous avons réutiliser le code le plus possible. 
Ce qui implique que le traitement est le même, la seule différence liée au format est l'endroit où l'on applique ce traitement.
Nous travaillons après le header dans le cas des bitmaps alors que pour les gifs, nous traitons les local color table.
Cela se justifie par le fait que celles-ci ne souffrent pas de compression et garantissent, en principe, un décodage sans encombres.\\

Pour terminer, nous avons également envisagé d'utiliser une autre technique appelée le MSB. 
Comme son nom l'indique, elle implique de travailler sur les bits de poids fort, ce qui trahit la manipulation du fichier. 
Ceci ne fait partie de notre implémentation que pour démontrer ce qu'il se passe lorqu'on ignore la bonne pratique.
 
\vspace {5cm}

\section {Références}
https://docs.microsoft.com/\\
https://www.fileformat.info/\\
http://www.matthewflickinger.com/lab/whatsinagif/\\